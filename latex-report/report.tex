\documentclass[12pt,a4paper]{draft}

% ============================================================
% Packages
% ============================================================
\usepackage[utf8]{inputenc}
\usepackage[T1]{fontenc}
\usepackage{geometry}
\usepackage{graphicx}
\usepackage{float}
\usepackage{listings}
\usepackage{xcolor}
\usepackage{hyperref}
\usepackage{booktabs}
\usepackage{enumitem}
\usepackage{fancyhdr}
\usepackage{titlesec}

% ============================================================
% Page Setup
% ============================================================
\geometry{margin=1in}
\pagestyle{fancy}
\fancyhf{}
\rhead{CS157C Final Project}
\lhead{Social Network Graph Application}
\rfoot{Page \thepage}

% ============================================================
% Code Listing Style
% ============================================================
\definecolor{codegreen}{rgb}{0,0.6,0}
\definecolor{codegray}{rgb}{0.5,0.5,0.5}
\definecolor{codepurple}{rgb}{0.58,0,0.82}
\definecolor{backcolour}{rgb}{0.95,0.95,0.92}

\lstdefinestyle{cypherStyle}{
    backgroundcolor=\color{backcolour},
    commentstyle=\color{codegreen},
    keywordstyle=\color{blue}\bfseries,
    numberstyle=\tiny\color{codegray},
    stringstyle=\color{codepurple},
    basicstyle=\ttfamily\footnotesize,
    breakatwhitespace=false,
    breaklines=true,
    captionpos=b,
    keepspaces=true,
    numbers=left,
    numbersep=5pt,
    showspaces=false,
    showstringspaces=false,
    showtabs=false,
    tabsize=2,
    frame=single,
    morekeywords={MATCH, RETURN, WHERE, CREATE, MERGE, DELETE, SET, WITH, ORDER, BY, LIMIT, AS, DISTINCT, NOT, AND, OR, OPTIONAL, UNWIND, count, type, SIZE}
}
\lstset{style=cypherStyle}

% ============================================================
% Title
% ============================================================
\title{
    \vspace{-1cm}
    \textbf{CS157C: NoSQL Database Systems}\\
    \Large Final Project Report\\
    \vspace{0.5cm}
    \huge Social Network Graph Application
}

\author{
    \textbf{Team Members}\\
    Aditya Dawadikar \texttt{<aditya.dawadikar@sjsu.edu>}\\
    Timothy \texttt{<timothy@sjsu.edu>}\\
    Jakob \texttt{<jakob@sjsu.edu>}
}

\date{December 2024}

% ============================================================
% Document Start
% ============================================================
\begin{document}

\maketitle
\thispagestyle{empty}
\newpage

\tableofcontents
\newpage

% ============================================================
% SECTION: Team Info
% ============================================================
\section{Team Information}

\begin{table}[H]
\centering
\begin{tabular}{@{}lll@{}}
\toprule
\textbf{Name} & \textbf{Email} & \textbf{Responsibilities} \\
\midrule
Aditya Dawadikar & aditya.dawadikar@sjsu.edu & Dataset, Schema, Search \& Explore \\
Timothy & timothy@sjsu.edu & User Management (UC-1 to UC-4) \\
Jakob & jakob@sjsu.edu & Social Graph Features (UC-5 to UC-9) \\
\bottomrule
\end{tabular}
\caption{Team Member Information}
\end{table}

% ============================================================
% SECTION: Property Graph Schema
% ============================================================
\section{Property Graph Schema}

\subsection{Node: User}

The \texttt{User} node represents a user in the social network.

\begin{table}[H]
\centering
\begin{tabular}{@{}lll@{}}
\toprule
\textbf{Property} & \textbf{Type} & \textbf{Description} \\
\midrule
userId & STRING & Zero-padded unique ID (0001...5000) \\
username & STRING & Generated unique username \\
email & STRING & User email address \\
name & STRING & Full display name \\
bio & STRING & Short profile biography \\
passwordHash & STRING & bcrypt hashed password \\
\bottomrule
\end{tabular}
\caption{User Node Properties}
\end{table}

\subsection{Relationship: FOLLOWS}

The \texttt{FOLLOWS} relationship represents a directed edge from User A to User B, indicating that User A follows User B.

\begin{verbatim}
(UserA:User)-[:FOLLOWS]->(UserB:User)
\end{verbatim}

% ============================================================
% SECTION: Dataset Information
% ============================================================
\section{Dataset Information}

\subsection{Dataset Source}
\begin{itemize}
    \item \textbf{Name:} Synthetic Social Network Dataset
    \item \textbf{Type:} Generated using Python scripts
    \item \textbf{Size:} 5,000 users, 15,000+ FOLLOWS relationships
\end{itemize}

\subsection{Dataset Description}
The dataset is synthetically generated to simulate a realistic social network with:
\begin{itemize}
    \item \textbf{User Generation:} 5,000 unique users with randomized usernames (Adjective + Noun + Number format)
    \item \textbf{Cluster Behavior:} Users are organized into 20 clusters for realistic community structure
    \item \textbf{Influencers:} 20 users with 1,000-3,500 followers each
    \item \textbf{Ghost Accounts:} 800 users with 0 followers (inactive accounts)
    \item \textbf{Normal Users:} Follow 3-50 users, primarily within their cluster
\end{itemize}

\subsection{Data Processing and Loading}

The data was generated and loaded using the following Python scripts:

\begin{enumerate}
    \item \texttt{generate\_users.py} - Creates 5,000 unique user records
    \item \texttt{generate\_graph.py} - Creates FOLLOWS relationships with realistic patterns
    \item \texttt{ingest\_graph.py} - Loads CSV data into Neo4j
\end{enumerate}

\subsection{Cypher Statements for Data Creation}

\textbf{Create Constraints:}
\begin{lstlisting}
CREATE CONSTRAINT user_id_unique IF NOT EXISTS
FOR (u:User) REQUIRE u.userId IS UNIQUE;

CREATE CONSTRAINT username_unique IF NOT EXISTS
FOR (u:User) REQUIRE u.username IS UNIQUE;
\end{lstlisting}

\textbf{Load Users:}
\begin{lstlisting}
UNWIND $rows AS row
MERGE (u:User {userId: row.userId})
SET u.username = row.username,
    u.email = row.email,
    u.name = row.name,
    u.bio = row.bio,
    u.passwordHash = row.passwordHash;
\end{lstlisting}

\textbf{Load FOLLOWS Relationships:}
\begin{lstlisting}
UNWIND $rows AS row
MATCH (f:User {userId: row.followerId})
MATCH (t:User {userId: row.followeeId})
MERGE (f)-[:FOLLOWS]->(t);
\end{lstlisting}

% ============================================================
% SECTION: Use Case Evidence
% ============================================================
\section{Use Case Evidence}

% ------------------------------------------------------------
% UC-1 through UC-4 (Timothy)
% ------------------------------------------------------------
\subsection{UC-1: User Registration}

\textbf{Screenshot:}
\begin{figure}[H]
    \centering
    \includegraphics[width=0.9\textwidth]{images/uc1_user_registration.png}
    \caption{UC-1: User Registration}
    \label{fig:uc1}
\end{figure}

\subsection{UC-2: User Login}

\textbf{Screenshot:}
\begin{figure}[H]
    \centering
    \includegraphics[width=0.9\textwidth]{images/uc2_user_login.png}
    \caption{UC-2: User Login}
    \label{fig:uc2}
\end{figure}

\subsection{UC-3: View Profile}

\textbf{Screenshot:}
\begin{figure}[H]
    \centering
    \includegraphics[width=0.9\textwidth]{images/uc3_view_profile.png}
    \caption{UC-3: View Profile}
    \label{fig:uc3}
\end{figure}

\subsection{UC-4: Edit Profile}

\textbf{Screenshot:}
\begin{figure}[H]
    \centering
    \includegraphics[width=0.9\textwidth]{images/uc4_edit_profile.png}
    \caption{UC-4: Edit Profile}
    \label{fig:uc4}
\end{figure}

% ------------------------------------------------------------
% UC-5: Follow Another User (Jakob)
% ------------------------------------------------------------
\subsection{UC-5: Follow Another User}

\textbf{Description:} A user can follow another user, creating a \texttt{FOLLOWS} relationship in Neo4j. The relationship is stored as a directed edge in the graph database, representing a one-way connection from the follower to the user being followed.

\textbf{Cypher Query:}
\begin{lstlisting}
// UC-5: Follow Another User
// Creates a FOLLOWS relationship between two users

MATCH (follower:User {userId: $followerId})
MATCH (target:User {userId: $targetId})
MERGE (follower)-[r:FOLLOWS]->(target)
RETURN 
    follower.userId AS followerId,
    follower.username AS followerUsername,
    target.userId AS targetId,
    target.username AS targetUsername,
    type(r) AS relationship
\end{lstlisting}

\textbf{Implementation Notes:}
\begin{itemize}
    \item Uses \texttt{MERGE} to prevent duplicate relationships
    \item Validates that a user cannot follow themselves
    \item Checks if relationship already exists before creating
    \item Returns confirmation of the created relationship
\end{itemize}

\textbf{Screenshot:}
\begin{figure}[H]
    \centering
    \includegraphics[width=0.9\textwidth]{images/uc5_follow_user.png}
    \caption{UC-5: Follow Another User - Creating a FOLLOWS relationship}
    \label{fig:uc5}
\end{figure}

% ------------------------------------------------------------
% UC-6: Unfollow a User (Jakob)
% ------------------------------------------------------------
\subsection{UC-6: Unfollow a User}

\textbf{Description:} A user can unfollow another user, removing the \texttt{FOLLOWS} relationship from the graph database. This operation deletes the directed edge between the two users.

\textbf{Cypher Query:}
\begin{lstlisting}
// UC-6: Unfollow a User
// Removes the FOLLOWS relationship between two users

MATCH (follower:User {userId: $followerId})
      -[r:FOLLOWS]->
      (target:User {userId: $targetId})
DELETE r
RETURN 
    follower.userId AS followerId,
    follower.username AS followerUsername,
    target.userId AS targetId,
    target.username AS targetUsername,
    'DELETED' AS status
\end{lstlisting}

\textbf{Implementation Notes:}
\begin{itemize}
    \item Uses pattern matching to find the exact relationship
    \item Validates relationship exists before deletion
    \item Only deletes the specific \texttt{FOLLOWS} edge, not the nodes
    \item Provides feedback on successful deletion
\end{itemize}

\textbf{Screenshot:}
\begin{figure}[H]
    \centering
    \includegraphics[width=0.9\textwidth]{images/uc6_unfollow_user.png}
    \caption{UC-6: Unfollow a User - Removing a FOLLOWS relationship}
    \label{fig:uc6}
\end{figure}

% ------------------------------------------------------------
% UC-7: View Friends/Connections (Jakob)
% ------------------------------------------------------------
\subsection{UC-7: View Friends/Connections}

\textbf{Description:} A user can see a list of people they are following (outgoing \texttt{FOLLOWS} relationships) and who follow them (incoming \texttt{FOLLOWS} relationships). This provides a complete view of a user's social connections.

\textbf{Cypher Query - View Followers:}
\begin{lstlisting}
// UC-7: View Followers
// Returns all users who follow the selected user

MATCH (u:User {userId: $userId})<-[:FOLLOWS]-(f:User)
RETURN 
    f.userId AS id,
    f.username AS username,
    f.name AS name,
    f.bio AS bio
ORDER BY f.username
LIMIT 100
\end{lstlisting}

\textbf{Cypher Query - View Following:}
\begin{lstlisting}
// UC-7: View Following
// Returns all users that the selected user follows

MATCH (u:User {userId: $userId})-[:FOLLOWS]->(t:User)
RETURN 
    t.userId AS id,
    t.username AS username,
    t.name AS name,
    t.bio AS bio
ORDER BY t.username
LIMIT 100
\end{lstlisting}

\textbf{Implementation Notes:}
\begin{itemize}
    \item Separates followers and following into distinct queries for clarity
    \item Uses directional relationship patterns (\texttt{<-[:FOLLOWS]-} vs \texttt{-[:FOLLOWS]->})
    \item Includes user details (name, bio) for display
    \item Limited to 100 results for performance
\end{itemize}

\textbf{Screenshots:}
\begin{figure}[H]
    \centering
    \includegraphics[width=0.9\textwidth]{images/uc7_view_followers.png}
    \caption{UC-7: View Followers - Users who follow the selected user}
    \label{fig:uc7_followers}
\end{figure}

\begin{figure}[H]
    \centering
    \includegraphics[width=0.9\textwidth]{images/uc7_view_following.png}
    \caption{UC-7: View Following - Users the selected user follows}
    \label{fig:uc7_following}
\end{figure}

% ------------------------------------------------------------
% UC-8: Mutual Connections (Jakob)
% ------------------------------------------------------------
\subsection{UC-8: Mutual Connections}

\textbf{Description:} A user can see mutual friends---users who are followed by both User A and User B. This feature demonstrates Neo4j's pattern matching capability to find intersection patterns in the graph.

\textbf{Cypher Query:}
\begin{lstlisting}
// UC-8: Mutual Connections
// Finds users that BOTH User A and User B follow
// Uses graph pattern matching to find intersection

MATCH (a:User {userId: $userAId})
      -[:FOLLOWS]->
      (mutual:User)
      <-[:FOLLOWS]-
      (b:User {userId: $userBId})
WHERE a <> b
RETURN DISTINCT
    mutual.userId AS id,
    mutual.username AS username,
    mutual.name AS name,
    mutual.bio AS bio
ORDER BY mutual.username
LIMIT 50
\end{lstlisting}

\textbf{Implementation Notes:}
\begin{itemize}
    \item Uses a single pattern to match both relationships simultaneously
    \item The pattern \texttt{(a)-[:FOLLOWS]->(mutual)<-[:FOLLOWS]-(b)} finds users followed by both A and B
    \item \texttt{WHERE a <> b} ensures we're comparing two different users
    \item \texttt{DISTINCT} prevents duplicate results
    \item Efficient graph traversal compared to SQL JOIN operations
\end{itemize}

\textbf{Screenshot:}
\begin{figure}[H]
    \centering
    \includegraphics[width=0.9\textwidth]{images/uc8_mutual_connections.png}
    \caption{UC-8: Mutual Connections - Users followed by both User A and User B}
    \label{fig:uc8}
\end{figure}

% ------------------------------------------------------------
% UC-9: Friend Recommendations (Jakob)
% ------------------------------------------------------------
\subsection{UC-9: Friend Recommendations}

\textbf{Description:} The system suggests new people to follow based on common connections using graph traversal queries. This implements a ``friends-of-friends'' algorithm that recommends users followed by people you follow, but whom you don't follow yet.

\textbf{Cypher Query:}
\begin{lstlisting}
// UC-9: Friend Recommendations
// Uses 2-hop graph traversal to find friends-of-friends
// Excludes users already followed and the user themselves
// Ranks by number of mutual connections

MATCH (u:User {userId: $userId})
      -[:FOLLOWS]->(friend)
      -[:FOLLOWS]->(recommended)
WHERE NOT (u)-[:FOLLOWS]->(recommended)
  AND u <> recommended
WITH recommended, count(DISTINCT friend) AS mutualCount
RETURN 
    recommended.userId AS id,
    recommended.username AS username,
    recommended.name AS name,
    recommended.bio AS bio,
    mutualCount
ORDER BY mutualCount DESC, recommended.username
LIMIT $limit
\end{lstlisting}

\textbf{Implementation Notes:}
\begin{itemize}
    \item \textbf{2-hop traversal:} The pattern finds users two steps away in the graph
    \item \textbf{Exclusion filter:} Ensures we don't recommend users already followed
    \item \textbf{Self-exclusion:} Prevents recommending the user to themselves
    \item \textbf{Ranking:} Results are ordered by \texttt{mutualCount}, giving priority to stronger recommendations
    \item \textbf{Aggregation:} Uses \texttt{count(DISTINCT friend)} to count mutual connections
    \item This query demonstrates the power of graph databases for social network analytics
\end{itemize}

\textbf{Screenshot:}
\begin{figure}[H]
    \centering
    \includegraphics[width=0.9\textwidth]{images/uc9_friend_recommendations.png}
    \caption{UC-9: Friend Recommendations - Suggested users based on mutual connections}
    \label{fig:uc9}
\end{figure}

% ------------------------------------------------------------
% UC-10 and UC-11 (Aditya's sections - placeholder)
% ------------------------------------------------------------
\subsection{UC-10: Search Users}
\textit{[Aditya's implementation]}

\textbf{Screenshot:}
\begin{figure}[H]
    \centering
    \includegraphics[width=0.9\textwidth]{images/uc10_search_users.png}
    \caption{UC-10: Search Users}
    \label{fig:uc10}
\end{figure}

\subsection{UC-11: Explore Popular Users}
\textit{[Aditya's implementation]}

\textbf{Screenshot:}
\begin{figure}[H]
    \centering
    \includegraphics[width=0.9\textwidth]{images/uc11_popular_users.png}
    \caption{UC-11: Explore Popular Users}
    \label{fig:uc11}
\end{figure}

% ============================================================
% END
% ============================================================
\end{document}
